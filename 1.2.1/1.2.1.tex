\documentclass[a4paper,14pt]{extarticle}
\usepackage[a4paper,top=1.3cm,bottom=2cm,left=1.5cm,right=1.5cm,marginparwidth=0.75cm]{geometry}
\usepackage{setspace}
\usepackage{cmap}					
\usepackage{mathtext} 				
\usepackage[T2A]{fontenc}			
\usepackage[utf8]{inputenc}			
\usepackage[english,russian]{babel}
\usepackage{multirow}
\usepackage{graphicx}
\usepackage{wrapfig}
\usepackage{tabularx}
\usepackage{float}
\usepackage{longtable}
\usepackage{hyperref}
\hypersetup{colorlinks=true,urlcolor=blue}
\usepackage[rgb]{xcolor}
\usepackage{amsmath,amsfonts,amssymb,amsthm,mathtools} 
\usepackage{icomma} 
\mathtoolsset{showonlyrefs=true}
\usepackage{euscript}
\usepackage{mathrsfs}

\DeclareMathOperator{\sgn}{\mathop{sgn}}
\newcommand*{\hm}[1]{#1\nobreak\discretionary{}
	{\hbox{$\mathsurround=0pt #1$}}{}}

\usepackage[T2A]{fontenc}
\usepackage[utf8]{inputenc}
\usepackage[english,russian]{babel}
\usepackage{amsmath,amsfonts,amsthm, mathtools}
\usepackage{amssymb}
\usepackage{icomma}
\usepackage{graphicx}
\usepackage{wrapfig}
\RequirePackage{longtable}

\usepackage{soulutf8} 
\usepackage{geometry}



\begin{document}
	\begin{center}
		\textit{Федеральное государственное автономное образовательное\\ учреждение высшего образования }
		
		\vspace{0.5ex}
		
		\textbf{«Московский физико-технический институт\\ (национальный исследовательский университет)»}
	\end{center}
	
	\vspace{10ex}
	
	
	\begin{center}
		\vspace{13ex}
		
		\textbf{Лабораторная работа №1.2.1}
		
		\vspace{1ex}
		
		по курсу общей физики
		
		на тему:
		
		\textbf{\textit{<<Определение скорости полёта пули при помощи баллистического маятника>>}}
		
		\vspace{30ex}
		
		\begin{flushright}
			\noindent
			\textit{Работу выполнил:}\\  
			\textit{Третьяков Александр \\(группа Б02-206)}
		\end{flushright}
		\vfill
		Долгопрудный \\ \today
		
		%\setcounter{page}{1}
	\end{center}
	
	\section{Аннотация}
	
	\textbf{Цель работы:}
	Проверить законы сохранения импульса и энергии.
	Определить скорость полёта пули, применяя законы сохранения и используя баллистические маятники.
	\newline Заявленная скорость вылета пули из духового ружья - 150-200 м/с.
	
	\noindent\textbf{Оборудование:}
	Духовое ружьё на штативе, осветитель, оптическая система для измерения отклонений маятника, измерительная линейка, пули и весы для их взвешивания, баллистические маятники.
	\newcommand{\RomanNumeralCaps}[1]
	{\MakeUppercase{\romannumeral #1}}
	\section{Ход работы}
	\paragraph{\Large\RomanNumeralCaps{1}\;\;\;\;{Метод баллистического маятника, совершающего поступательное движение}}
	\subsection{Методика измерений}
	В этой части работы будем использовать установку, изображённую на рисунке ниже. При попадании пули в цилиндр любая его точка движется по окружности известного радиуса, поэтому его смещение с помощью собирающей линзы можно перевести в линейное отклонение на линейке.
	
	\begin{figure}[H]
		\begin{center}
			\includegraphics[scale = 0.56]{"ustan_1.2.1_1.png"}
			\caption{схема установки для измерения скорости полета пули}
		\end{center}
	\end{figure}
	Для начала проверим работоспособность установки, а именно проведем несколько холостых выстрелов по маятнику и убедимся в том, что он практически не реагирует на удар воздушной струи.
	Вычислять скорость пули будем по форуле \eqref{vel1}, для чего нужно проверить, что за 10 колебаний амплитуда уменьшалается меньше, чем на половину.
	Произведем 4 выстрела, запишем амплитуды, полученные при выстрелах, и по их значениям найдем скорости пуль.

	\subsection{Теоретические сведения}
	\begin{center}
		При контакте пули с цилиндром можно записать ЗСИ:
		\begin{equation}
			mu = (M+m)V \Leftrightarrow u=\frac{M+m}{m}V \approx \frac{M}{m}V 
		\end{equation}
		где $m$ -- масса пули, $u$ -- скорость пули перед ударом, $V$-- скорость цилиндра вместе с пулей после удара.
 	\end{center}
    Из ЗСЭ выразим скорость цилиндра после соударения с пулей:
	\begin{equation}
		 V^2=2gh \;\;\;\;\; h = L(1-cos \varphi ) = 2L sin \frac{\varphi^2}{2}, \;\;\;\;\text{где}\;\;\; \varphi \approx \frac{\Delta x}{L},\;\;\;\; \text{таким образом}\;\;\; V^2=\frac{g\Delta x^2}{L}
	\end{equation}
	Тогда скорость пули можно выразить как
	\begin{equation} \label{vel1}
		u=\frac{M}{m} \sqrt{\frac{g}{L}} \Delta x
	\end{equation}
	\subsection{Используемое оборудование}
	Массу пули (m) измеряем при помощи весов - погрешность: $\sigma_m = 0,005$г.
	\\При помощи оптической системы измеряем линейную амплитуду колебаний маятника ($\Delta x$); -- погрешность измерений - цена деления линейки: $\sigma_{\Delta x} = c = 0,1$мм.
	\\При помощи большой линейки измеряем длину нитей маятника (L); -- погрешность измерения нельзя считать равной цене деления линейки ввиду человечекого фактора (измерения происходят на глаз, так как маятник подвешен к потолку): $\sigma_L = 1$см.
	\\Масса цилиндра баллистического маятника была измерена лаборантом: $\sigma_M = 5$г.
	\subsection{Результаты измерений} 	
	\begin{table}[H]
		\begin{center}
			\begin{tabular}[H]{|c|c|c|c|c|}
				\hline
				$m$, г & 0,499 & 0,504 & 0,503 & 0,502\\ \hline
				$\Delta x$, мм & 9,5 & 9,8 & 9,2 & 10,4\\ \hline
				$u$, м/с & 118,1 & 120,6 & 113,4 & 128,5 \\ \hline
			\end{tabular}
			\caption{}
		\end{center}
	\end{table}
	Наша установка имела параметры: $M = (2925 \pm 5)$ г, и $L = (218 \pm 1)$ см.
	Средняя скорость пули \underline{$u_\text{ср} = 120,2$ м/с}, а погрешность будет равна:
	
	
		$$\sigma_u^{\text{сист}} =u \sqrt{\varepsilon_M^2 + \varepsilon_m^2 + \varepsilon_{\Delta x}^2 + \left(\frac{\varepsilon_L}{2} \right)^2}\;\;\;\;\;\; \sigma_u^\text{сист}\approx 1,8  \text{ }\dfrac{\text{м}}{\text{с}}$$  \;\;\;\;\;$$ \sigma_u^{\text{случ}} = \sqrt{ \frac{1}{n(n-1)} \sum_{i=1}^{n}(u_i - u_{\text{ср}})^2}\;\;\;\;\;\;\;\; \sigma_u^\text{случ}\approx 3,2 \text{ }\dfrac{\text{м}}{\text{с}}$$ \;\;\;\;\;$$ \sigma_u =\sqrt{\sigma_{\text{сист}}^2 + \sigma_\text{случ}^2}\;\;\;\;\;\;\sigma_u \approx 3,7 \text{ }\dfrac{\text{м}}{\text{с}}$$
	
	
	Окончательно получаем скорость пули равную \underline{$u = (120,2 \pm 3,7)\text{, }\dfrac{\text{м}}{\text{с}}$}
	
	\paragraph{\Large\RomanNumeralCaps{2}\;\;\;\;{Метод крутильного баллистического маятника}}
	\subsection{Методика измерений}
	В этой части работы мы будем использовать крутильный баллистический маятник. Схема установки представлена на картинке ниже.
	
	\begin{figure}[h]
		\begin{center}
			\includegraphics[scale = 0.66]{"ustan_1.2.1_2.png"}
			\caption{схема установки для измерения скорости полета пули с баллистическим маятником}
		\end{center}
	\end{figure}
	\subsection{Теоретическая справка}
	Считая удар неупругим, можно записать уравнение:
	$$mur=I \Omega$$
	,где $r-$расстояние от линии полёта пули до оси вращения, $I$ -- момент инерции относительно этой оси, $\Omega$ -- угловая скорость маятника сразу после удара.
	\\Можно пренебречь затуханием колебаний и потерями энергии и записать ЗСЭ:
	$$ k \frac{\varphi^2}{2} = I \frac{\Omega^2}{2} $$
	\noindent ,где $k$ -- модуль кручения проволоки, $\varphi$ -- максимальный угол поворота маятника, тогда:
	\begin{equation} \label{vel2}
		u = \varphi \frac{\sqrt{kI}}{mr} 
	\end{equation}
	Измерим растояние от оси вращения до штатива с линейкой $d = 50,6 \pm 0,1 \text{ см}$, тогда в силу малости колебаний можно найти $\varphi$ как
	
	\begin{equation}
		\label{phi}
		\varphi \approx \frac{x}{2d}
	\end{equation}
	,где $x$ -- смещение изображения нити осветителя на шкале, которое легко можно измерить.
	\\Периоды колебаний маятника с грузами и без можно выразить как:
	$$T_1= 2 \pi \sqrt{\frac{I}{k}} \;\;\;\;\;\; T_2 = 2 \pi \sqrt{\frac{I - 2MR^2}{k}}$$
	Тогда $\sqrt{kI}$ можно найти как:
	\begin{equation}
		\sqrt{kI} = \frac{4 \pi M R^2 T_1}{T_1^2 - T_2^2}
		\label{kl}
	\end{equation}
	$R$ -- расстояние от оси вращения до центров грузиков, $M$ - масса грузиков.
	\section{Результаты измерений}
	Для начала запишем данные установки: $$ r = 21,5 \text{ см} \text{, } R = 33,9 \text{ см} \text{, } M_1 = 729,9\text{ г} \text{, а } M_2 = 729,6 \text{ г} $$.
	Снимем периоды колебаний после выстрела с грузиками и без, чтобы найти $\sqrt{kI}$:
	
	\begin{table}[H]
		\begin{center}
			\begin{tabular}{|c|c|c|c|}
				\hline
				№  & $t$, с  & $T$, с&  N \\
				\hline
				1. С грузами & 158 & 19,75 & 8\\
				\hline
				2. Без грузиков & 118 & 14,75& 8\\
				\hline 
			\end{tabular}
			\caption{Периоды колебаний баллистического маятника после выстрела}
		\end{center}
	\end{table}
	
	Из таблицы получаем, что $T_1^{\text{ср}} = (19,75 \pm 0,125) \text{ см}$, а $T_2^{\text{ср}} = (14,75 \pm 0,125)\text{ см}$. С помощью полученных периодов колебаний найдем $\sqrt{kI}$ по формуле \eqref{kl}:
	
	$$\sqrt{kI} \approx 120,65 \cdot 10^{-3} \text{ } \dfrac{\text{кг}\cdot\text{м}^2}{\text{c}}$$ \;\;\;\;\;\; $$\sigma_{\sqrt{kI}} = \sqrt{kI} \cdot \sqrt{\varepsilon_{T_2^2-T_1^2}^2 + \left(2\varepsilon_{R^2}\right)^2 + \varepsilon_M^2 + \varepsilon_{T^2}^2} \approx 0,12 \cdot 10^{-3} \text{ } \dfrac{\text{кг}\cdot\text{м}^2}{\text{c}}$$
	
	Теперь по формулам \eqref{vel2} и \eqref{phi} определим $\varphi$ и скорость пули. Получаем таблицу:
	
	\begin{table}[!h]
		\centering
			\begin{tabular}{|c|c|c|c|c|}
				\hline
				& $m$, г& $x$, cм&  $\varphi$, рад& $u$, м/с\\
				\hline
				1 & 0,503 & 12,25 &0,12 & 133,86\\
				\hline
				2 &0,516  &13,70  & 0,136 & 147,90 \\
				\hline
				3 & 0,499 &13,50  &0,133  &149,57 \\
				\hline
				4 & 0,507 &13,80  & 0,136 & 150,53\\
				\hline 
			\end{tabular}
			\caption{Таблица полученных скоростей}
	\end{table}
	$$\sigma_u^{\text{сист}} = u\cdot \sqrt{ \varepsilon_x^2+ \varepsilon_d^2+ \varepsilon_{\sqrt{kI}}^2 + \varepsilon_m^2 + \varepsilon_r^2 } \;\;\;\;\;\; \sigma_u^{\text{случ}} =  \sqrt{\frac{1}{n(n-1)} \sum_{i=1}^{n}(u_i - \overline{u})^2}$$ \;\;\;\;\;\;
	$$\sigma_u = \sqrt{\sigma_{\text{случ}}^2 + \sigma_\text{сист}^2}
	$$
	
	Тогда средняя скорость \underline{$u_\text{ср} = (145,47 \pm 1,44)\text{ }\frac{\text{м}}{\text{с}} $}
	
	\section{Вывод}
	
	\indent Были полученны скорости пули двумя методами:  методом баллистического маятника, совершающего поступательное движение, и методом крутильного баллистического маятника. Различие полученных значений с табличными может быть связано со истраченным баллоном в ружье. Так же имеет значение то, что стрельба в каждом методе производилась своим ружьем.
	
\end{document}