\documentclass[a4paper,14pt]{extarticle}
% Преамбула создана для онлайн-смены "Абитуриент 2022"
% Автор - Баринов Леонид, telegram: @leonidka2001
%
% pdflatex
%
% Кодировка. Существуюет входная и внутренняя кодировка.
% Мы будем пользоваться внутренней кодировкой T2A и входной
% кодировкой utf8. Стоит понимать, что это большой костыль!
% Если создаете новые собственные файлы используйте другие 
% версия TeXa, которые хорошо работают с разными языками и
% различными математическими шрифтами, например, xelatex!
\usepackage[T2A]{fontenc}
\usepackage[utf8]{inputenc}


% Для соблюдения типографских традиций и возможности переноса
% слов в различных языках нужен пакет babel. Языки указывается через запятую,
% основный язык документа указывается последним

\usepackage[english,russian]{babel}


%-----------------------------------------------------
%% xelatex
%
%% Если используете xelatex, переключите компилятор на xelatex
%% закомментируйте строчки с fontenc, inputenc, babel и amssymb
%% раскомментируйте строчки ниже до следующей линии ----
%
%% Вместо babel
%\usepackage{polyglossia}   
%\setdefaultlanguage{russian}  
%\setotherlanguage{english} 
%
%% Подключение математических символов
%\usepackage{unicode-math}
%
%\setmainfont{CMU Serif} %% задаёт основной шрифт документа
%\setsansfont{CMU Sans Serif} %% задаёт шрифт без засечек
%\setmonofont{CMU Typewriter Text} %% задаёт моноширинный шрифт
%\setmathfont{latinmodern-math.otf}
%\setmathfont[range={\rightarrow,\leftarrow,\int,\vartriangle, \mitg}]{XITS Math}
%
%% Возможность использования русских букв в математическом
%% режиме без команды \text{text}
%
%\DeclareSymbolFont{cyrletters}{\encodingdefault}{\familydefault}{m}{}
%\newcommand{\makecyrmathletter}[1]{%
%	\begingroup\lccode`a=#1\lowercase{\endgroup
%		\Umathcode`a}="0 \csname symcyrletters\endcsname\space #1
%}
%\count255="409
%\loop\ifnum\count255<"44F
%\advance\count255 by 1
%\makecyrmathletter{\count255}
%\repeat

%------------------------------------------------------

% Работа с математическими символами, добавление
% различных математических окружений.
% Спасибо Американскому математическому обществу!

\usepackage{amsmath,amsfonts,amsthm, mathtools}
\usepackage{amssymb}

% Вставка рисунков. Можно указать место, где необходимо
% искать изображения

\usepackage{graphicx}
\graphicspath{{images/}}

% Вставка плавающих объектов (занимающих часть страницы)
\usepackage{wrapfig}

% latex вставляет рисунки по определенному алгоритму. Его,
% конечно, можно менять, но это не настолько просто. Как
% правило, хочется, чтобы картинка располагалась там, где мы это
% указали в коде. Для этого существует несколько пакетов, один из
% них floatrow. Он позволяет для окружения figure указывать
% необязательный аргумент - H (именно большое h), что на latex'овском
% языке означает: вставить картинку здесь и только здесь. (даже если
% облик документа несколько пострадает)

\usepackage{floatrow}

% По правилам оформления рисунок всегда должен быть подписан. Для
% этого существует команда \caption{}. Но обычные настройки caption
% оставлять желать лучшего. Хотелось сделать подпись меньше
% основного шрифта, а также слово Рис жирным и использовать
% разделитель точку, а не двоеточие. В этом помогает пакет,
% который называется caption (совпадение?)

\usepackage{setspace}
\usepackage[margin=10pt,font={small,stretch=0.9},labelfont=bf,labelsep=period,
justification=centerlast]{caption}

% Оформление и создание таблиц 

\usepackage{array,tabularx,tabulary,booktabs}

% Великолепный новый пакет для работы с таблицами
% \usepackage{tabularray}

% После excel есть ощущения, что везде объединить колонки или строки
% легко. В latex не совсем так. Помогают пакеты multirow, multicol. 

\RequirePackage{multirow}
\RequirePackage{multicol}

% Иногда могут потребоваться длинные таблицы на несколько страниц.
% Обычные таблицы latex воспринимает как одну букву. И
% становиться понятно, почему возникают проблемы при переносе
% обычной таблицы. (Ведь нельзя же перенести одну букву!). Поэтому
% вместо обычной таблицы нужна длинная таблица.

\RequirePackage{longtable}

% В русской типографской традиции принято начинать каждый новый абзац
% с красной строки. Даже первый после заголовка (или подзаголовка).
% Чтобы каждый раз не ставить красную строку вручную существует пакет
% indentfirst

\usepackage{indentfirst}


% Работа со ссыллаками:
\usepackage{hyperref}

% Целая дробная часть у нас разделяется запятой,
% однако так принято не во всём мире. Чтобы TeX
% воспринимал запятую как разделитель целой и дробной
% части необходим пакет icomma

\usepackage{icomma}

%% Перенос знаков в формулах (по Львовскому)
\newcommand*{\hm}[1]{#1\nobreak\discretionary{}
	{\hbox{$\mathsurround=0pt #1$}}{}}

% Модификаторы начертания
\usepackage{soulutf8} 

%%% Программирование
\usepackage{etoolbox} % логические операторы

% Работа с колонтитулами
\usepackage{fancyhdr}
%\pagestyle{fancy}
\renewcommand{\headrulewidth}{0mm} % Если необходимо убрать линейку, или изменить ее длину
% \lfoot{\thepage} % Нижний левый
\cfoot{\thepage} % Нижний в центре
% \rfoot{} % Нижний правый
% \rhead{} % Верхний правый
% \chead{} % Верхний в центре
% \lhead{} % Верхний левый



% Задание полей документа. Есть несколько способов, но
% самый простой из них - это воспользоваться пакетом geometry, который
% позволяет определить все поля документа (начиная с краев листа, что
% важно, так как некоторые другие способы позволяют это сделать только косвенно)

\usepackage{geometry}
\geometry{top=20mm}
\geometry{bottom=20mm}
\geometry{left=20mm}
\geometry{right=20mm}

% Свои команды
\DeclareMathOperator{\sgn}{\mathop{sgn}}



\begin{document}
Первый         абзац.

Второй абзац.
$2 +    2 =   4$. Текст абзаца.
\[  2+2=4  \]

$2,4$, \hspace{1em} $2, 4$

\vspace{1em}

Текст текст текст текст текст текст текст текст текст текст текст $1\hm{+}2+3+4+5+6=21$

\begin{equation}
	3+3 = 6
\end{equation}

\begin{equation}
	a^2 + b^2 = c^2
	\label{eq:pif}
\end{equation}


\eqref{eq:pif}  на стр. \pageref{eq:pif} --- теорема Пифагора.

Закон Гука:
\[
F_{\text{упр}} = kx, \hspace{1em} \beta = 30^\circ
\]

\section{Нюансы работы с формулами}

\subsection{Дроби}

\[ \frac{1}{2} \hspace{1em} \frac{1+\dfrac{4}{2}}{6} = 0,5\]

\subsection{Скобки}

\[ \left\{2+\frac{9}{3} + \frac{1+\dfrac{4}{2}}{6}\right\} \times 5 = 25 \% \$ \]

\[  [2+3]  \]

\[ \{2+3\}  \]

\subsection{Стандартные функции}

$\sin x = 0$, $\cos x = 1$, $\ln x = 5$

$\sgn  x = 1$

\subsection{Символы}

$2\cdot 2 \ne 5$

$A \cap B$, $A \cup B$

\subsection{Диакритические знаки}

$\overline{456789xyz}=5$, $\widetilde{eurhkjs7} = 8$

$\vec{asdfv} + \vec b = \vec c$, $\overrightarrow{AB}$


\subsection{Буквы других алфавитов}

$\tg \Gamma = 1$

$\epsilon$, $\phi$

$\varepsilon$, $\varphi$

\renewcommand{\epsilon}{\varepsilon}
\renewcommand{\phi}{\varphi}
$\epsilon, \phi$

\section{Формулы в несколько строк}

\subsection{Очень длинная формула}

\begin{multline}
	1+ 2+3+4+5+6+7+\dots + \\ 
	+ 50+51+52+53+54+55+56+57 + \dots + \\ 
	+ 96+97+98+99+100=5050 \tag{S} \label{eq:sum}
\end{multline}


\subsection{Несколько формул}
\begin{align}
	&2\times 2 = 4 & 6\times 8 &= 48 \\
	&3\times 3 = 9 & a+b &= c\\
	&10 \times 65464 = 654640 & 3/2&=1,5
\end{align}

\begin{equation}
	\begin{aligned}
		2\times 2 &= 4 & 6\times 8 &= 48 \\
		3\times 3 &= 9 & a+b &= c\\
		10 \times 65464 &= 654640 & 3/2&=1,5
	\end{aligned}
\end{equation}
\newpage
\[2\times 2 = 4, \hspace{1em} 3\times 3 = 9,\]
\[10 \times 65464 = 654640\]
\[2=2\]
\begin{gather}
	2\times 2 = 4, \hspace{1em} 3\times 3 = 9,\\[0.2cm]
	10 \times 65464 = 654640\\
	2=2
\end{gather}

\begin{equation}
\begin{gathered}
	2\times 2 = 4, \hspace{1em} 3\times 3 = 9,\\
	10 \times 65464 = 654640
\end{gathered}
\end{equation}

\subsection{Системы уравнений}

\[
\left\{
\begin{aligned}
	&2\times x = 4  \\
	&3\times y = 9\\
	&10 \times 65464 = z\\
\end{aligned}
\right.
\]

\[
|x|=\begin{cases}
	x, &\text{если}\;  x \ge 0 \\
	-x, &\text{если } x<0
\end{cases}
\]

\section{Матрицы}

\[
\begin{pmatrix}
	a_{11} & a_{12} & a_{13} \\
	a_{21} & a_{22} & a_{23} \\
	a_{31} & a_{32} & a_{33}
\end{pmatrix}
\]

\[
\begin{vmatrix}
	a_{11} & a_{12} & a_{13} \\
	a_{21} & a_{22} & a_{23}
\end{vmatrix}
\]

\[
\begin{bmatrix}
	a_{11} & a_{12} & a_{13} \\
	a_{21} & a_{22} & a_{23}
\end{bmatrix}
\]

В уравнении \eqref{eq:sum} на стр. \pageref{eq:sum} много слагаемых.

\subsubsection*{Парочка замечаний}

В \textbf{результате} решения получаем ответ

\[\mathbf x =\frac{-1+\frac12}
{3+\left(\cfrac{5}{11}\right)^{12}}\]



\[ f(x)=5x \eqno \text{(номер)} \] 

\begin{align*}
	sfd \notin \\
	\equiv \\
\end{align*}

\noindent Распространенная ошибка, распространенная ошибка:

\[ f(x) = x + 5x + x^2\]
Отступы не симметричны, отступы не симметричны




\section{Рисунки}
\subsection{Растровые рисунки}

\includegraphics[scale=0.3]{znak.png}\\

\includegraphics[scale=0.7]{znak.png}\\

\includegraphics[width=15cm,height=7cm,keepaspectratio]{znak.png}\\

\includegraphics[draft]{znak.png}

\subsection{Векторные картинки}

\includegraphics[width=\textwidth]{logo.pdf}

\section{Таблицы}

\begin{tabular}{|m{8cm}|c||m{5cm}|}
	\hline
	Это очень-очень длинное предложение из многих слов. & $6\times 2$ &  Это очень-очень длинное предложение из многих слов. \\ 
	\hline
	21 & \rule{0cm}{0.8cm} $\displaystyle \frac{x}{y}$ & 23  \\[0.4cm]
	\hline
	31 & 32 & 33 \\ 
	\hline 
\end{tabular} 
\vspace{1em}

\begin{tabularx}{\textwidth}{X|c|X}
	\hline
	Это очень-очень длинное предложение из многих слов & Текст покороче & Это очень-очень длинное предложение из многих слов Это очень-очень длинное предложение из многих слов
\end{tabularx}


\begin{tabulary}{\textwidth}{C|J|R}
	\hline
	Это очень-очень длинное предложение из многих слов & Текст покороче & Это очень-очень длинное предложение из многих слов Это очень-очень длинное предложение из многих слов
\end{tabulary}

\section{Плавающие объекты}

Смотри таблицу \ref{tab:mytab}.

\begin{table}[H]
	\begin{center}
		\caption[Заголовок для списка таблиц]{Бессмысленная таблица, зато с кучей фишек.}\label{tab:mytab}
		\begin{tabular}{|c|c|c|c||l|c|c|r|c|c|}
			\hline
			1 & 2 & 3 & 4 & 5 & 6 & 7 & 8 & 9 & 10 \\ \hline
			Первый & Второй & \multicolumn{3}{|c|}{Третий -- пятый} &   &  & Восьмой &   &  \\ 
			\cline{1-7} \cline{9-10}
			1 & 2 & 3 & 4 & 5 & 6 & 7 & 8 & 9 & 10 \\ \hline \hline
			1 & 2 & 3 & 4 & 5 & 6 & 7 & 8 & 9 & 10 \\ \hline
			\multirow{3}{*}{Три строки}  & 2 & 3 & 4 & 5 & 6 & 7 & 8 & 9 & 10 \\ \cline{2-10}
			& 2 & 3 & 4 & 5 & 6 & 7 & 8 & 9 & 10 \\ \cline{2-10}
			& 2 & 3 & 4 & 5 & 6 & 7 & 8 & 9 & 10 \\ \hline
		\end{tabular}
	\end{center}
	%\caption{Заголовок мог быть и здесь}
\end{table}




\begin{longtable}{|c|c|c|c|}
	\caption{Заголовок большой таблицы.}\\
	\hline
	\textbf{RND1} & \textbf{RND2} & \textbf{RND3} & \textbf{RND4} \\ \hline
	\endfirsthead
	\hline
	RND1 & RND2 & RND3 & RND4 \\ \hline
	\endhead
	\hline
	\multicolumn{4}{r}{продолжение следует\ldots} \
	\endfoot
	\hline
	\endlastfoot
	
	0,576745371 & 0,435853468 & 0,36384912 & 0,299047979 \\ 
	0,064795364 & 0,028454613 & 0,751312059 & 0,693972684 \\
	0,263563971 & 0,367508634 & 0,075536384 & 0,337780707 \\
	0,957583964 & 0,431948588 & 0,938522377 & 0,464307785 \\
	0,815740484 & 0,123129806 & 0,883432767 & 0,760983283 \\
	0,445062335 & 0,157424268 & 0,883442259 & 0,300596338 \\
	0,187159669 & 0,728663343 & 0,637199982 & 0,765684528 \\
	0,41009848 & 0,457031472 & 0,142858106 & 0,602946607 \\
	0,43315663 & 0,26058316 & 0,611667007 & 0,400328185 \\
	0,824086963 & 0,27304335 & 0,244565296 & 0,219675484 \\
	0,109578811 & 0,278478018 & 0,242519359 & 0,414669471 \\
	0,220778432 & 0,938106645 & 0,502630894 & 0,910760406 \\
	0,905239004 & 0,017835419 & 0,429423867 & 0,299079986 \\
	0,604679988 & 0,784786124 & 0,86825382 & 0,003631105 \\
	0,725883239 & 0,273875543 & 0,843605984 & 0,607743466 \\
	0,555736787 & 0,019487901 & 0,342950631 & 0,537183422 \\
	0,309374962 & 0,44331087 & 0,749656403 & 0,966836051 \\
	0,274332831 & 0,740197878 & 0,865450742 & 0,792816484 \\
	0,968626843 & 0,580215733 & 0,706427331 & 0,879562225 \\
	0,281344607 & 0,51362826 & 0,7998827 & 0,270290356 \\
	0,885143961 & 0,989455756 & 0,235591368 & 0,693434397 \\
	0,505067377 & 0,127308502 & 0,614625825 & 0,277375342 \\
	0,663594497 & 0,023550761 & 0,670822594 & 0,302446663 \\
	0,094723947 & 0,091199224 & 0,841117852 & 0,617394243 \\
	0,490246305 & 0,761569651 & 0,973576975 & 0,51597127 \\
	0,631301873 & 0,155944248 & 0,319958965 & 0,198643097 \\
	0,853761692 & 0,993889567 & 0,105045533 & 0,837805396 \\
	0,149834425 & 0,316419619 & 0,387770251 & 0,552013475 \\
	0,269182006 & 0,721020214 & 0,484218147 & 0,552132834 \\
	0,668632873 & 0,699511389 & 0,278877959 & 0,021775345 \\
	0,62638369 & 0,737702261 & 0,696351048 & 0,256427487 \\
	0,922563692 & 0,629514529 & 0,789891184 & 0,019748079 \\
	0,366649518 & 0,882085214 & 0,805771543 & 0,461659364 \\
	0,178967822 & 0,400706498 & 0,313063544 & 0,425676173 \\
	0,328582166 & 0,124008134 & 0,177734655 & 0,653821253 \\
	0,318628436 & 0,924056157 & 0,005170407 & 0,09988244 \\
	0,1523348 & 0,686022531 & 0,877786704 & 0,230997696 \\
	0,160048577 & 0,475334591 & 0,118018156 & 0,720594848 \\
	0,502602506 & 0,898504748 & 0,103602236 & 0,289059862 \\
	0,185262766 & 0,640333509 & 0,980932923 & 0,424269289 \\
	0,63740761 & 0,665837647 & 0,256564927 & 0,796877433 \\
	0,326795292 & 0,863892719 & 0,19537989 & 0,410369904 \\
	0,377332846 & 0,61459335 & 0,158101373 & 0,100684292 \\
	0,540188499 & 0,911708617 & 0,077277867 & 0,108818241 \\
	0,485200234 & 0,692007154 & 0,012528805 & 0,364692863 \\
	0,435947515 & 0,555444136 & 0,410076838 & 0,973027822 \\
	0,423053661 & 0,502696027 & 0,500150945 & 0,209929767 \\
	0,146604488 & 0,318962234 & 0,535025906 & 0,25597358 \\
	0,252933039 & 0,897587117 & 0,961039174 & 0,238301151 \\
	0,798559806 & 0,885674601 & 0,451623639 & 0,903044881 \\
	0,467795852 & 0,398491485 & 0,09863235 & 0,110588673 \\
	0,932456386 & 0,679931054 & 0,499049066 & 0,419347908 \\
	0,806742814 & 0,998944815 & 0,730738513 & 0,207088322 \\
	0,524028453 & 0,251332909 & 0,711910448 & 0,243583774 \\
	0,037417208 & 0,333822686 & 0,276647434 & 0,882818666 \\
	0,358649112 & 0,534662608 & 0,726203191 & 0,041117785 \\
	0,141309914 & 0,36643456 & 0,552053605 & 0,956487966 \\
	0,53808496 & 0,939874695 & 0,186724749 & 0,690302117 \\
	0,052101497 & 0,887611776 & 0,677925016 & 0,622234766 \\
	0,553154653 & 0,040281685 & 0,504952332 & 0,097544063 \\
	0,732288281 & 0,658739311 & 0,883348524 & 0,144957902 \\
	0,288649747 & 0,517727905 & 0,639432157 & 0,456739615 \\
	0,293369191 & 0,138002629 & 0,154228354 & 0,133189564 \\
	0,693221668 & 0,246693033 & 0,465542044 & 0,978720597 \\
	0,135587928 & 0,15068455 & 0,825417066 & 0,885949167 \\
	0,676052335 & 0,253724745 & 0,219361854 & 0,808580891 \\
	0,582461065 & 0,554730526 & 0,476287005 & 0,268673107 \\
	0,238129516 & 0,090469211 & 0,525167086 & 0,59620778 \\
	0,769704124 & 0,27036399 & 0,888763617 & 0,089602751 \\
	0,548435183 & 0,357753532 & 0,858061896 & 0,465681708 \\
	0,702731358 & 0,856923488 & 0,058935386 & 0,675796794 \\
	0,338117119 & 0,622858325 & 0,461848295 & 0,94572588 \\
	0,606619551 & 0,999527337 & 0,361750308 & 0,673771858 \\
	0,221137745 & 0,719189979 & 0,624447286 & 0,59032258 \\
	0,239784727 & 0,636404041 & 0,841898027 & 0,844823258 \\
	0,800614467 & 0,368896918 & 0,994129014 & 0,291457496 \\
	0,681757552 & 0,019367985 & 0,417601531 & 0,649347809 \\
	0,28051889 & 0,061635488 & 0,914332594 & 0,331713964 \\
	0,657743996 & 0,983965656 & 0,818946725 & 0,36394332 \\
	0,543479307 & 0,169289586 & 0,483196672 & 0,985172369 \\
	0,145081556 & 0,892455096 & 0,190462767 & 0,824433551 \\
	0,196973955 & 0,995308839 & 0,879891823 & 0,845636911 \\
	0,904947195 & 0,593928658 & 0,403422613 & 0,076252813 \\
	0,269580321 & 0,740772576 & 0,182364329 & 0,695081896 \\
	0,293711052 & 0,351494187 & 0,331350034 & 0,62158188 \\
	0,69779066 & 0,019424915 & 0,657473072 & 0,783698296 \\
	0,14204222 & 0,817006985 & 0,669234791 & 0,728306309 \\
	0,38941124 & 0,807135743 & 0,702842593 & 0,382494957 \\
	0,203543688 & 0,969191131 & 0,822881425 & 0,212473701 \\
	0,826623142 & 0,181291269 & 0,054701556 & 0,386442059 \\
	0,541365118 & 0,573617788 & 0,650112336 & 0,930417614 \\
	0,277453725 & 0,382833978 & 0,395547164 & 0,785051981 \\
	0,078149646 & 0,115526198 & 0,417197235 & 0,894812516 \\
	0,772854891 & 0,698024923 & 0,504995217 & 0,492422679 \\
	0,592288285 & 0,153957871 & 0,348784682 & 0,523821625 \\
	0,618156868 & 0,841905787 & 0,038053593 & 0,861496223 \\
	0,76387049 & 0,652733723 & 0,034948244 & 0,814496925 \\
\end{longtable}

\begin{wrapfigure}[8]{l}{0.4\linewidth}
	\includegraphics[width=\linewidth]{logo}
	\caption{Картинка с обтеканием}
\end{wrapfigure}Московский физико-технический институт — ведущий технический вуз страны, который входит в престижные рейтинги лучших университетов мира. Здесь обучают фундаментальной и прикладной физике, математике, информатике, химии, биологии, компьютерным технологиям и другим естественным и точным наукам. Сегодня Физтех - это передовой научный центр. За последние годы здесь были открыты 64 новые лаборатории, где работают ученые с мировым именем. Они занимаются проблемами старения и возрастных заболеваний, прикладной и фундаментальной физикой, нанооптикой, квантовыми вычислениями, фотоникой и многим другим.



День рождения Физтеха —  25 ноября. В этот день в 1946 году вышло постановление Совета Министров СССР о создании физико-технического факультета (ФТФ) МГУ, а через 5 лет на его базе был создан Московский физико-технический институт.



Отцами-основателями Физтеха считаются три нобелевских лауреата: Пётр Капица, Николай Семёнов и Лев Ландау, а также проректор МГУ по специальным вопросам и куратор предшественника МФТИ ФТФ МГУ Сергей Христианович.

\renewcommand{\arraystretch}{1.2}
\begin{wraptable}[10]{r}{0.5\linewidth}
	\vspace{-0.6cm}
	\begin{tabular}{|m{4.5cm}|c|}
		\hline
		Национальные рейтинги & В России \\ \hline
		Мониторинг качества приёма в ВУЗы & 1 \\ \hline
		<<Эксперт РА>> (RAEX) & 2 \\ \hline
		<<Интерфакс>> & 3 \\ \hline
		Forbes Россия & 3 \\ \hline
	\end{tabular}
	\caption{Обтекаемая таблица}
\end{wraptable}Они заложили основу образования в МФТИ — уникальную «систему Физтеха»: отбор самых талантливых абитуриентов и вовлечение студентов в реальную научно-исследовательскую работу под руководством выдающихся ученых в базовых организациях. Таких ор\-га\-низаций-партнеров на Физтехе больше ста. Среди них, помимо целого ряда институтов Российской академии наук, крупнейшие российские компании: Яндекс, Сбербанк-технологии, ABBYY и многие другие.



В МФТИ ежегодно учатся около 7 000 студентов, а преподают более 80 академиков и членов-корреспондентов РАН. Кстати, это дает самое большое число академиков на одного студента среди всех российских вузов.




За время существования из МФТИ выпустилось более 36 тысяч человек, многие из которых добились успехов в самых разных областях науки, бизнеса и даже искусства. 150 из них стали академиками и членами-корреспондентами РАН,а один - Президентом РАН. 6000 получили звание  доктора наук, 17000 стали кандидатами наук. Физтех дал миру двух нобелевских лауреатов —  Андрея Гейма и Константина Новоселова, трех летчиков-космонавтов, трех министров по науке и одного из авторов архитектурных принципов построения вычислительных комплексов Бориса Бабаяна.


\begin{figure}[H]
% \CenterFloatBoxes
% \TopFloatBoxes
% \BottomFloatBoxes
	\begin{floatrow}
		\floatbox[\captop]{table}[0.3\textwidth]{\caption{Table} }{
			\begin{tabular}{|c|c|}
				\hline
				kslfj & lkjdf \\ \hline
				jdf; & dkjlj \\ \hline
			\end{tabular}
		}
		\floatbox[\captop]{table}[0.3\textwidth]{\caption{My Caption two}}{
			\begin{tabular}{|c|c|}
				\hline
				kslfj & lkjdf \\ \hline
				jdf; & dkjlj \\ \hline
				sdlkf & dlkf\\ \hline
			\end{tabular}
		}
	\end{floatrow}
\end{figure}

\begin{figure}[H]
	 \CenterFloatBoxes
	% \TopFloatBoxes
	% \BottomFloatBoxes
	\begin{floatrow}
		\floatbox{figure}[0.3\textwidth]{\caption{МФТИ} }{
			\includegraphics[width=\linewidth]{znak}
		}
		\floatbox{table}[0.4\textwidth]{\caption{My Caption two}}{
			\begin{tabular}{|c|c|}
				\hline
				kslfj & lkjdf \\ \hline
				jdf; & dkjlj \\ \hline
				sdlkf & dlkf\\ \hline
				sdlkf & dlkf\\ \hline
				sdlkf & dlkf\\ \hline
				sdlkf & dlkf\\ \hline
			\end{tabular}
		}
	\end{floatrow}
\end{figure}


\newpage
\section*{Рисунок}

На рис.~\ref{squares} изображены два квадрата: красный и черный.

\begin{figure}[H]
	\centering
	\includegraphics[scale=0.2]{figure}
	\caption{Квадраты}\label{squares}
\end{figure}

Прекрасный рисунок!


%%% Теоремы
\theoremstyle{plain} % Это стиль по умолчанию, его можно не переопределять.
\newtheorem{theorem}{Теорема}[section]
\newtheorem{proposition}[theorem]{Утверждение}

\theoremstyle{definition} % "Определение"
\newtheorem{corollary}{Следствие}[theorem]
\newtheorem{problem}{Задача}[section]

\theoremstyle{remark} % "Примечание"
\newtheorem*{nonum}{Решение}

\section{Теоремы}

\begin{theorem}[Простое равенство]\label{theorem1}
	$2+2=4$
\end{theorem}

Смотри теорему \ref{theorem1} на стр. \pageref{theorem1}.

\begin{proposition}
	$3\times 3 = 9$
\end{proposition}

\begin{corollary}
	$9/3 = 3$
\end{corollary}

\begin{nonum}
	Всё что угодно.
\end{nonum}

\section{Новые команды}

\newcommand{\nw}[1]{\int\limits_0^1 #1 dx}

\begin{equation}\label{pref}
	\nw{x^2}
\end{equation}

Вычислите интеграл $\nw{x^3}$

\newcommand{\str}[1]{%
	на стр. \pageref{#1}%
}

\newcommand{\myint}[2][x]{
	\begin{equation}
		\int\limits_0^1 #2 d #1
	\end{equation}
}

\myint{\frac{1}{x}}

$x\ge y$

\renewcommand{\ge}{\geqslant}

$x\ge y$

\section{Счетчики}

\newcounter{nc}[section]
\arabic{nc}

\setcounter{nc}{5}
\arabic{nc}

\Roman{nc}

\alph{nc}

\Asbuk{nc}

\arabic{section}

\renewcommand{\thesection}{\Asbuk{section}}
\newcommand{\z}[1]{%
	\addtocounter{nc}{1}
	Задача \thesection.\arabic{nc}. #1%
}

\z{Текст задачи.}

\z{Problem two}

\setcounter{section}{0}

\section{Окружения}

\newenvironment{zadacha}[1]{
	\addtocounter{nc}{1}%
	Задача \thesection.\arabic{nc}. <<#1>> %
}{\vspace{1cm}}

\begin{zadacha}{Главная задача}
	Текст задачи.
\end{zadacha}

\section{etoolbox}

\newbool{answers}
\booltrue{answers}
\renewcommand{\z}[2][Нет ответа.]{%
	\vspace{1mm}
	\addtocounter{nc}{1}
	\noindent Задача \thesection.\arabic{nc}. #2\\%
	\ifbool{answers}{Ответ. #1}{}
}

\z[6]{Сколько будет $2+4$?}

\z{Сколько будет $2+ \infty $?}

\renewcommand{\thesection}{\arabic{section}}
\setcounter{section}{8}
\section{Кегль}

\begin{table}[H]
	\centering
	\begin{tabular}{|c|c|}
		\hline	\verb|\tiny|      & \tiny        крошечный \\
		\hline	\verb|\scriptsize|   & \scriptsize  очень маленький\\
		\hline \verb|\footnotesize| & \footnotesize  довольно маленький \\
		\hline \verb|\small|        &  \small        маленький \\
		\hline \verb|\normalsize|   &  \normalsize  нормальный \\
		\hline \verb|\large|        &  \large       большой \\
		\hline \verb|\Large|        &  \Large       еще больше \\[5pt]
		\hline \verb|\LARGE|        &  \LARGE       очень большой \\[5pt]
		\hline \verb|\huge|         &  \huge        огромный \\[5pt]
		\hline \verb|\Huge|         &  \Huge        громадный \\ \hline
	\end{tabular}
\caption{Размеры шрифта}
\end{table}

\begin{Huge}
	\textbf{Какой-нибудь \textit{обычный}  текст.}
\end{Huge}

Какой-нибудь \emph{текст \emph{с} выделением}.
\vfill
\section{Титульный лист}
\newpage

\thispagestyle{empty}
\begin{center}
	\textit{Федеральное государственное автономное образовательное\\ учреждение высшего образования }
	
	\vspace{0.5ex}
	
	\textbf{«Московский физико-технический институт\\ (национальный исследовательский университет)»}
\end{center}

\vspace{10ex}

\begin{center}
	\vspace{13ex}
	
	\textbf{Лабораторная работа №2.1.}
	
	\vspace{1ex}
	
	по курсу общей физики
	
	на тему:
	
	\textbf{\textit{<<Опыт Франка-Герца>>}}
	
	\vspace{30ex}
	
	\begin{flushright}
		\noindent
		\textit{Работу выполнил:}\\  
		\textit{Баринов Леонид \\(группа Б02-827)}
	\end{flushright}
	\vfill
	Долгопрудный \\ \today
	
	%\setcounter{page}{1}
\end{center}
\newpage

\section{Гиперссылки}

Сайт кубка ЛФИ \url{https://lpr-olimp.ru/}

Сайт \href{https://mipt.ru/}{МФТИ}


\section{Перечни}

\begin{itemize}
	\item[*] Первый пункт 
	\item 
	\begin{itemize}
		\item Вложенный список
		\item ляляля 
		\begin{enumerate}
			\item Первый пункт нумерованного списка
			\item Второй пункт
		\end{enumerate}
	\end{itemize}
	\item Второй пункт
\end{itemize}

\begin{enumerate}
	\begin{multicols}{2}
		\item $(1+x_1)(1+x_2)^2$;
		\item $\sqrt{x_1}$;
		\item $x_1+2x_2-10$;
		\item $(0{,}5x_1+x_2)^2$;
		\item $x_2$;
		\item $\sqrt{x_1}+\sqrt{2x_2}$;
		\item $\ln (1+x_1)+2\ln(1+x_2)$;
		\item $5x_1$;
		\item $10-x_1+2x_2$.
	\end{multicols}
\end{enumerate}

\section{Отображение счетчиков}


\setcounter{nc}{5}

Счетчик установлен на значении 4
\begin{center}
	
	\begin{tabular}{|cc|}
		\hline Команда & Представление \\ \hline
		\verb|\arabic|  & \arabic{nc} \\ \hline
		\verb|\roman|  & \roman{nc} \\ \hline
		\verb|\Roman|  & \Roman{nc} \\ \hline
		\verb|\alph|  & \alph{nc} \\ \hline
		\verb|\Alph|  & \Alph{nc} \\ \hline
		\verb|\asbuk|  & \asbuk{nc} \\ \hline
		\verb|\Asbuk|  & \Asbuk{nc} \\ \hline
		\verb|\fnsymbol|  & \fnsymbol{nc} \\ \hline
	\end{tabular}
\end{center}
\newpage

\section{Форматы текста}

Некоторые из форматов требуют подключения пакета soulutf8.

\begin{center}
\begin{tabular}{|cc|}
	\hline Команда & Представление \\ \hline
	\verb|\textmd|  & \textmd{Слово | Word} \\ 
	\verb|\textsc|  & \textsc{Слово | Word} \\ 
	\verb|\textup|  & \textup{Слово | Word} \\ 
	\verb|\textit|  & \textit{Слово | Word} \\ 
	\verb|\textsc|  & \textsc{Слово | Word} \\ 
	\verb|\textsl|  & \textsl{Слово | Word} \\ 
	\verb|\textrm|  & \textrm{Слово | Word} \\ 
	\verb|\textsf|  & \textsf{Слово | Word} \\ 
	\verb|\texttt|  & \texttt{Слово | Word} \\ 
	\verb|\textbf|  & \textbf{Слово | Word} \\ 
	\verb|\underline|  & \underline{Слово | Word} \\ 
	\verb|\textbf\textit|  & \textbf{\textit{Слово | Word}} \\   \hline
\end{tabular}
\end{center}


\end{document}