\documentclass[a4paper,14pt]{extarticle}
% Преамбула создана для онлайн-смены "Абитуриент 2022"
% Автор - Баринов Леонид, telegram: @leonidka2001
%
% pdflatex
%
% Кодировка. Существуюет входная и внутренняя кодировка.
% Мы будем пользоваться внутренней кодировкой T2A и входной
% кодировкой utf8. Стоит понимать, что это большой костыль!
% Если создаете новые собственные файлы используйте другие 
% версия TeXa, которые хорошо работают с разными языками и
% различными математическими шрифтами, например, xelatex!
\usepackage[T2A]{fontenc}
\usepackage[utf8]{inputenc}


% Для соблюдения типографских традиций и возможности переноса
% слов в различных языках нужен пакет babel. Языки указывается через запятую,
% основный язык документа указывается последним

\usepackage[english,russian]{babel}


%-----------------------------------------------------
%% xelatex
%
%% Если используете xelatex, переключите компилятор на xelatex
%% закомментируйте строчки с fontenc, inputenc, babel и amssymb
%% раскомментируйте строчки ниже до следующей линии ----
%
%% Вместо babel
%\usepackage{polyglossia}   
%\setdefaultlanguage{russian}  
%\setotherlanguage{english} 
%
%% Подключение математических символов
%\usepackage{unicode-math}
%
%\setmainfont{CMU Serif} %% задаёт основной шрифт документа
%\setsansfont{CMU Sans Serif} %% задаёт шрифт без засечек
%\setmonofont{CMU Typewriter Text} %% задаёт моноширинный шрифт
%\setmathfont{latinmodern-math.otf}
%\setmathfont[range={\rightarrow,\leftarrow,\int,\vartriangle, \mitg}]{XITS Math}
%
%% Возможность использования русских букв в математическом
%% режиме без команды \text{text}
%
%\DeclareSymbolFont{cyrletters}{\encodingdefault}{\familydefault}{m}{}
%\newcommand{\makecyrmathletter}[1]{%
	%	\begingroup\lccode`a=#1\lowercase{\endgroup
		%		\Umathcode`a}="0 \csname symcyrletters\endcsname\space #1
	%}
%\count255="409
%\loop\ifnum\count255<"44F
%\advance\count255 by 1
%\makecyrmathletter{\count255}
%\repeat

%------------------------------------------------------

% Работа с математическими символами, добавление
% различных математических окружений.
% Спасибо Американскому математическому обществу!

\usepackage{amsmath,amsfonts,amsthm, mathtools}
\usepackage{amssymb}

% Вставка рисунков. Можно указать место, где необходимо
% искать изображения

\usepackage{graphicx}
\graphicspath{{images/}}

% Вставка плавающих объектов (занимающих часть страницы)
\usepackage{wrapfig}

% latex вставляет рисунки по определенному алгоритму. Его,
% конечно, можно менять, но это не настолько просто. Как
% правило, хочется, чтобы картинка располагалась там, где мы это
% указали в коде. Для этого существует несколько пакетов, один из
% них floatrow. Он позволяет для окружения figure указывать
% необязательный аргумент - H (именно большое h), что на latex'овском
% языке означает: вставить картинку здесь и только здесь. (даже если
% облик документа несколько пострадает)

\usepackage{floatrow}

% По правилам оформления рисунок всегда должен быть подписан. Для
% этого существует команда \caption{}. Но обычные настройки caption
% оставлять желать лучшего. Хотелось сделать подпись меньше
% основного шрифта, а также слово Рис жирным и использовать
% разделитель точку, а не двоеточие. В этом помогает пакет,
% который называется caption (совпадение?)

\usepackage{setspace}
\usepackage[margin=10pt,font={small,stretch=0.9},labelfont=bf,labelsep=period,
justification=centerlast]{caption}

% Оформление и создание таблиц 

\usepackage{array,tabularx,tabulary,booktabs}

% Великолепный новый пакет для работы с таблицами
% \usepackage{tabularray}

% После excel есть ощущения, что везде объединить колонки или строки
% легко. В latex не совсем так. Помогают пакеты multirow, multicol. 

\RequirePackage{multirow}
\RequirePackage{multicol}

% Иногда могут потребоваться длинные таблицы на несколько страниц.
% Обычные таблицы latex воспринимает как одну букву. И
% становиться понятно, почему возникают проблемы при переносе
% обычной таблицы. (Ведь нельзя же перенести одну букву!). Поэтому
% вместо обычной таблицы нужна длинная таблица.

\RequirePackage{longtable}

% В русской типографской традиции принято начинать каждый новый абзац
% с красной строки. Даже первый после заголовка (или подзаголовка).
% Чтобы каждый раз не ставить красную строку вручную существует пакет
% indentfirst

\usepackage{indentfirst}


% Работа со ссыллаками:
\usepackage{hyperref}

% Целая дробная часть у нас разделяется запятой,
% однако так принято не во всём мире. Чтобы TeX
% воспринимал запятую как разделитель целой и дробной
% части необходим пакет icomma

\usepackage{icomma}

%% Перенос знаков в формулах (по Львовскому)
\newcommand*{\hm}[1]{#1\nobreak\discretionary{}
	{\hbox{$\mathsurround=0pt #1$}}{}}

% Модификаторы начертания
\usepackage{soulutf8} 

%%% Программирование
\usepackage{etoolbox} % логические операторы

% Работа с колонтитулами
\usepackage{fancyhdr}
%\pagestyle{fancy}
\renewcommand{\headrulewidth}{0mm} % Если необходимо убрать линейку, или изменить ее длину
% \lfoot{\thepage} % Нижний левый
\cfoot{\thepage} % Нижний в центре
% \rfoot{} % Нижний правый
% \rhead{} % Верхний правый
% \chead{} % Верхний в центре
% \lhead{} % Верхний левый



% Задание полей документа. Есть несколько способов, но
% самый простой из них - это воспользоваться пакетом geometry, который
% позволяет определить все поля документа (начиная с краев листа, что
% важно, так как некоторые другие способы позволяют это сделать только косвенно)

\usepackage{geometry}
\geometry{top=20mm}
\geometry{bottom=20mm}
\geometry{left=20mm}
\geometry{right=20mm}

% Свои команды
\DeclareMathOperator{\sgn}{\mathop{sgn}}